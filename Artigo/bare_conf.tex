\documentclass[conference]{IEEEtran}
\usepackage[utf8]{inputenc}

%\usepackage[portuguese]{babel}
%
%\usepackage{glossaries}  
%\setacronymstyle{short-long}
%
%\newacronym{mpls}{MPLS}{Multiprotocol Label Switching}
%\newacronym{asic}{ASIC}{Application-specific Integrated Circuit}
%
%\makeglossaries

\usepackage[pdftex]{graphicx}
\graphicspath{{./img/}}

% correct bad hyphenation here
\hyphenation{op-tical net-works semi-conduc-tor DEFCON}
%

\begin{document}
\title{Wireless Personal Area Networks\\
  \large Redes de Comunicações Móveis\\
  2016/2017
}

\author{\IEEEauthorblockN{Fábio Teixeira}
\IEEEauthorblockA{Faculdade de Ciências\\da\\Universidade do Porto\\
Email: up*********@fc.up.pt}
\and
\IEEEauthorblockN{Vanessa Silva}
\IEEEauthorblockA{Faculdade de Ciências\\da\\Universidade do Porto\\
Email: up201305731@fc.up.pt}
}

% make the title area
\maketitle

% As a general rule, do not put math, special symbols or citations
% in the abstract
\begin{abstract}
Resumo aqui.
\end{abstract}


\IEEEpeerreviewmaketitle


\section{Introdução}
Introdução aqui.

\section{Conceito, Motivação, Objetivos, Benefícios}

\subsection{Conceito}

Uma \textit{Wireless Personal Area Network} (WPAN) é uma rede de área pessoal (\textit{Personal Area Network (PAN)}) que permite conectar dispositivos centrados na área (curta) de uma pessoa, com base em conexões sem fio (wireless). Deste modo, WPAN forma uma "bolha" wireless em torno de uma pessoa, conhecida como \textit{Personal Operating Space} (POS) \cite{prasad2004ofdm}, e essa bolha pode, dinamicamente, expandir-se e contrair-se consoante as necessidades.

Além da conexão entre os dispositivos pessoais, que formam a bolha, WPAN deve fornecer ao utilizador uma conexão \textit{ad hoc} (\ref{redes_ad_hoc}) com os recursos e aplicações compatíveis que entram no seu POS \cite{prasad2004ofdm}.
Esta permite que os dispositivos pessoas se comuniquem com outros dispositivos em que a faixa de comunicação interceta o seu POS.

O paradigma das comunicações tradicionais destina-se a estabelecer ligações de comunicação entre dispositivos, enquanto que as WPANs destinam-se a estabelecer comunicações entre pessoas, (dispositivos pessoais habilitados para comunicação), e recursos funcionais/dados através de um meio sem fios (substituindo o tradicional "cabo").




\subsection{Motivação}

A motivação para tais redes vem da necessidade de troca de dados não só em grandes distâncias (que é tradicionalmente referida como comunicações), mas também entre pessoas que estão em uma "conversação" em uma curta distância, assim como da necessidade dessa troca de dados ser realizada em um meio sem fios.


\subsection{Objetivos}

Os principais objetivos das WPANs são \cite{prasad2004ofdm}:

\begin{itemize}
 \item Baixo consumo de energia - problema crítico, dado que a velocidade com que o desempenho da bateria melhora é bastante lenta, em comparação com o explosivo crescimento das comunicações sem fio;
 \item Operação no espectro não licenciado - WPANs usam ligações sem fio não licenciadas, uma vez que esta é a única forma de conseguirem conectividade onipresente sem impacto adverso em uma infraestrutura sem fio existente;
 \item Baixo custo;
 \item Pequeno tamanho de pacote.
\end{itemize}

\subsection{Benefícios}

%Livro:  A conectividade ad hoc proporcionada pelas WPANs pode motivar o design dos próprios dispositivos de computação, bem como a distribuição das tarefas e capacidades de computação em diferentes dispositivos.


\section{Funcionamento, Hardware}

\subsection{Funcionamento}

\subsection{Hardware - Tecnologias Associadas}

\subsection{Redes \textit{Ad Hoc}}
\label{redes_ad_hoc}


\section{Modelos de Arquitetura/\textit{standard}}

\subsection{Bluetooth}

\subsection{IrDA}

\subsection{IEEE 802.15.4 - Zigbee}

\subsection{HomeRF}

\subsection{UWB - Ultra-wideband}


%Exemplo de imagem em latex em duas colunas
%\begin{figure}[!t]
%  \centering
%  \includegraphics[width=0.45\textwidth]{switch_openflow.png}
%  \caption{Switch OpenFlow dedicado \cite{xia2015survey}.}
%  \label{fig:refswitchOP}
%\end{figure}

%Exemplo de referencia à imagem
%\ref{fig:refswitchOP}


\section{Conclusões}
Conclusões aqui.

\bibliography{referencias}{}
\bibliographystyle{IEEEtran}

\end{document}
