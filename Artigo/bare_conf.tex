\documentclass[conference]{IEEEtran}
\usepackage[utf8]{inputenc}

%\usepackage[portuguese]{babel}
%
%\usepackage{glossaries}  
%\setacronymstyle{short-long}
%
%\newacronym{mpls}{MPLS}{Multiprotocol Label Switching}
%\newacronym{asic}{ASIC}{Application-specific Integrated Circuit}
%
%\makeglossaries

\usepackage[pdftex]{graphicx}
\graphicspath{{./img/}}

% correct bad hyphenation here
\hyphenation{op-tical net-works semi-conduc-tor DEFCON}
%

\begin{document}
\title{Wireless Personal Area Networks\\
  \large Redes de Comunicações Móveis\\
  2016/2017
}

\author{\IEEEauthorblockN{Fábio Teixeira}
\IEEEauthorblockA{Faculdade de Ciências\\da\\Universidade do Porto\\
Email: up*********@fc.up.pt}
\and
\IEEEauthorblockN{Vanessa Silva}
\IEEEauthorblockA{Faculdade de Ciências\\da\\Universidade do Porto\\
Email: up201305731@fc.up.pt}
}

% make the title area
\maketitle

% As a general rule, do not put math, special symbols or citations
% in the abstract
\begin{abstract}
Resumo aqui.
\end{abstract}


\IEEEpeerreviewmaketitle


\section{Introdução}
Introdução aqui.

\section{Desenvolvimento aqui}

%Exemplo de imagem em latex em duas colunas
%\begin{figure}[!t]
%  \centering
%  \includegraphics[width=0.45\textwidth]{switch_openflow.png}
%  \caption{Switch OpenFlow dedicado \cite{xia2015survey}.}
%  \label{fig:refswitchOP}
%\end{figure}

%Exemplo de referencia à imagem
%\ref{fig:refswitchOP}


\section{Conclusões}
Conclusões aqui.

\bibliography{referencias}{}
\bibliographystyle{IEEEtran}

\end{document}
